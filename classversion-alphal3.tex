\NeedsTeXFormat{LaTeX2e}
\ProvidesPackage{classversion-alpha}[2022/02/03 v0.2.0 Used for 
versioning classes]
\RequirePackage{xparse}
\RequirePackage{pgf}
\RequirePackage{readprov}
\RequirePackage{alphalph}
\RequirePackage{ifthen}

% Typesetting commands
\providecommand*{\file}[1]{\texttt {#1}}
\providecommand*{\Lpack}[1]{\textsf {#1}}

\def\classname#1{\gdef\@classname{#1} \title{Installation check and 
release information for 
\Lpack{\@classname} up to release \classversion@currentRelease}}


%Counts number of releases
\ExplSyntaxOn
\int_new:c l_cver_releasenumber_int
\int_zero:cN l_cver_releasenumber_int
%Counts number of files - reset by new release
\int_new:c l_cver_releasenumber_int
\int_zero:cN l_cver_releasenumber_int
%Scratch counters
\int_new:c l_cver_int
\int_new:c l_cver_iint

%This preamble command signifies the beginning of a new release
\NewDocumentCommand{\NewRelease}{m m O{\relax}}{
    \gdef\classversion@currentRelease{#1} %This provides the current 
    %release version number
    
    \int_incr:c l_cver_releasenumber_int 
    
    %We define a command to return the release version number given a 
    %lower case letter representing the release number
    \cs_new:cpx 
    {cver_release\alphalph{\theclassversion@release}_tl} {#1}
    
    %And the release date
    \cs_new:cpx 
    {cver_date\alphalph{\theclassversion@release}_tl} {#2}
    
    %And the change summary
    \cs_new:cpx 
    {cver_summary\alphalph{\theclassversion@release}_tl} {#3}
    
    %We reset this counter to zero
    \int_zero:c l_cver_filenumber_int
    
    %We define a counter for this release to count its files and set it 
    %to zero
    \int_new:c 
    {l_cver_filesinrelease\alphalph{\theclassversion@release}}
    \int_zero:c 
    {l_cver_filesinrelease\alphalph{\theclassversion@release}}
}

%This converts a release number to a release version number
\NewDocumentCommand{\classversion@thisreleasenumber}{m}
    {\csname cver_release\alphalph{#1}_tl\endcsname}
%This converts a release number to a release date
\NewDocumentCommand{\classversion@thisreleasedate}{m}
    {\csname cver_date\alphalph{#1}_tl\endcsname}
%This converts a release number to a change summary
\NewDocumentCommand{\classversion@thisreleasesummary}{m}
    {\csname cver_summary\alphalph{#1}_tl\endcsname} 
    
    
%This returns the number of files in a release given the release number
\NewDocumentCommand{\classversion@filesinthisrelease}{m}
    {\csname theclassversion@filesinrelease\alphalph{#1} \endcsname}
\cs_new:cpn {cver_filesinthisrelease:n} #1
    {\int_show:c {l_cver_filesinrelase\alphalph{#1}}

%This preamble command follows \NewRelease and signifies a file that is 
%part of that release
\NewDocumentCommand{\IncludeFileInRelease}{m m}{
    \int_incr:c l_cver_filenumber_int
    
    \cs_new:cpx 
    {cver_file\AlphAlph{\theclassversion@release}\alphalph{\theclassversion@file}_tl}
    {#1}
    
    \cs_new:cpx 
    {cver_eversion\AlphAlph{\theclassversion@release}\alphalph{\theclassversion@file}_tl}
    {\ignorespaces#2\ignorespaces}
    
    \bool_new:c
        {cver_exists\AlphAlph{\theclassversion@release}\alphalph{\theclassversion@file}_bool}
    
    \IfFileExists{#1}{
        \ReadFileInfos{#1}
    
        \cs_new:cpx 
        {cver_version\AlphAlph{\theclassversion@release}\alphalph{\theclassversion@file}_tl}
        {\ignorespaces\UseVersionOf{#1}\ignorespaces}
    
        \cs_new:cpx 
        {cver_date\AlphAlph{\theclassversion@release}\alphalph{\theclassversion@file}_tl}
        {\UseDateOf{#1}}
        
        \bool_set_true:c
            {cver_exists\AlphAlph{\theclassversion@release}\alphalph{\theclassversion@file}_bool}
    }{
        \cs_new:cpn 
        {cver_version\AlphAlph{\theclassversion@release}\alphalph{\theclassversion@file}_tl}
        {\textcolor{red}{version\space missing}}
        
        \cs_new:cpn 
        {cver_date\AlphAlph{\theclassversion@release}\alphalph{\theclassversion@file}_tl}
        {\textcolor{red}{date\space missing}}
        
        \bool_set_false:c
        {cver_exists\AlphAlph{\theclassversion@release}\alphalph{\theclassversion@file}_tl}
    }
      
    \stepcounter{classversion@filesinrelease\alphalph{\theclassversion@release}}
}

%This prints a section to check the installation compared to the latest 
%release
\NewDocumentCommand{\installation}{O{Installation check}}{
    \section*{#1}
    This installation of \textsf{\@classname} 
    consists of the following 
    \classversion@filesinthisrelease{\theclassversion@release}\space 
    files:
    \begingroup
    \int_zero:c l_cver_int
    \begin{enumerate}
        \int_do_while:nn 
        {l_cver_int<\classversion@filesinthisrelease{\theclassversion@release}}
        {   \item
            \int_incr:c l_cver_int
            \file{\csname 
            cver_file\AlphAlph{\theclassversion@release}\alphalph{\theclassversion@int}_tl\endcsname}~
            \csname 
            cver_version\AlphAlph{\theclassversion@release}\alphalph{\theclassversion@int}_tl\endcsname,
             dated 
            \csname 
            cver_date\AlphAlph{\theclassversion@release}\alphalph{\theclassversion@int}_tl\endcsname
            
            {\color{red}(expecting \csname 
            cver_eversion\AlphAlph{\theclassversion@release}\alphalph{\theclassversion@int}_tl\endcsname)}
        }
    \end{enumerate}
    \endgroup
}
\ExplSyntaxOff
%This prints a section to show latest release
\NewDocumentCommand{\latestrelease}{O{Release manifest}}{
    \section*{#1}
    This release (\classversion@currentRelease) of \Lpack{\@classname} 
    consists of the following 
    \classversion@filesinthisrelease{\theclassversion@release}\space 
    files:
    \begingroup
    \setcounter{classversion@int}{0}
    \begin{enumerate}
        \loop
            \item
            \stepcounter{classversion@int}
            \file{\csname file\AlphAlph{\theclassversion@release}
            \alphalph{\theclassversion@int}\endcsname}~
            \csname eversion\AlphAlph{\theclassversion@release}
            \alphalph{\theclassversion@int}\endcsname
        \ifnum\theclassversion@int<\classversion@filesinthisrelease{\theclassversion@release}
         
        \repeat
    \end{enumerate}
    \endgroup
}

%This prints a section to show a specific release
\NewDocumentCommand{\thisrelease}{O{Release 
\classversion@thisreleasenumber{#2}, dated 
\classversion@thisreleasedate{#2}} m}{
    \section*{#1}
    \paragraph{Change summary}
    \classversion@thisreleasesummary{#2}
    \paragraph*{Manifest}
    This release consists of the following 
    \ifnum\classversion@filesinthisrelease{#2}>1 
    \classversion@filesinthisrelease{#2}\space files\else file\fi:
    \begingroup
    \setcounter{classversion@int}{0}
    \begin{enumerate}
        \loop
        \item
        \stepcounter{classversion@int}
        \file{\csname         
        file\AlphAlph{#2}\alphalph{\theclassversion@int}\endcsname}~\csname
         eversion\AlphAlph{#2}\alphalph{\theclassversion@int}\endcsname
        \ifnum\theclassversion@int<
            \classversion@filesinthisrelease{#2}
        \repeat
    \end{enumerate}
    \endgroup
}

%This prints a section to show each release, starting with the most 
%recent
\NewDocumentCommand{\releasehistory}{}{
    \begingroup
    \setcounter{classversion@iint}{\theclassversion@release}
    \loop
        \thisrelease{\theclassversion@iint}
        \addtocounter{classversion@iint}{-1}
    \ifnum\theclassversion@iint>0
    \repeat
    \endgroup
}

% \changes{v0.1.0}{2022/02/03}{Initial version}
% \changes{v0.2.0}{2022/02/04}{Refactored to expl}